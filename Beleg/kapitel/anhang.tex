%--------------------------------------------------------------------------------------
% kapitel/anhang.tex
%--------------------------------------------------------------------------------------
\appendix
\addtocontents{toc}{\protect\contentsline{chapter}{\appendixname}{}{}}
%--------------------------------------------------------------------------------------
% Anhang
%--------------------------------------------------------------------------------------
	\chapter{Motordrehung}
		\begin{center}
			\captionsetup{type=figure}
			\subfloat[\textbf{Schritt 1}: A-GND, /A-VCC, B-GND, /B-VCC ]{\resizebox{0.45\textwidth}{!}{\input{bilder/grafik_schrittmotor_3.pdf_tex}}}
			\quad
			\subfloat[\textbf{Schritt 2}: A-VCC, /A-GND,  B-GND, /B-VCC]{\resizebox{0.45\textwidth}{!}{\input{bilder/grafik_schrittmotor_4.pdf_tex}}}
			\\
			\subfloat[\textbf{Schritt 3}: A-VCC, /A-GND, B-VCC, /B-GND]{\resizebox{0.45\textwidth}{!}{\input{bilder/grafik_schrittmotor_1.pdf_tex}}}
			\quad
			\subfloat[\textbf{Schritt 4}: A-GND, /A-VCC, B-VCC, /B-GND]{\resizebox{0.45\textwidth}{!}{\input{bilder/grafik_schrittmotor_2.pdf_tex}}}
			\caption{Rotordrehung}
			\label{fig:motordrehung_1}		
		\end{center}
		
		
\chapter{Dateiverzeichnis der CD}

\begin{itemize}
	\item	\textbf{Bachelorarbeit} \\
	CD:\textbackslash	Bachelorarbeit.pdf
	\item \textbf{C-Quellcode im AVR Studio 6} \\
		CD:\textbackslash C-Quellcode AVR Studio 6\textbackslash StepControl\textbackslash StepControl.atsln
	\item	\textbf{Schaltungsdesign und Leiterplattenlayout: Altium Designer 10} \\
		CD:\textbackslash Schaltungsdesign, Leiterplattenlayout\textbackslash Schrittmotosteuerung\\
				\textbackslash Schrittmotosteuerung.PrjPcb
	\item	\textbf{Datenbl�tter:}\\
		CD:\textbackslash Datenbl�tter\textbackslash 
		\begin{itemize}
			\item ADC Widerstand
			\item Elektrolytkondensator
			\item Linearregler
			\item Logik-IC
			\item Mikrocontroller
			\item Pegelwandler
			\item Quartz
			\item	Schrittmotoren
			\item Motortreiber
		\end{itemize}
\end{itemize}

% Auch hier sind Gliederungen aller \chapter, \section
%\chapter{Auflistung von Quellcode und �hnliches}
%Eine M�glichkeit Quelltext aufzulisten.
%
%\lstinputlisting{listings/Hello.c}
%%--------------------------------------------------------------------------------------
%% firstline - Gibt die erste Zeile die dargestellez werden soll
%% lastline - letzte Zeile die aus der Quelltextdatei entnohmen, und dargestellt werden soll
%%--------------------------------------------------------------------------------------
%Und eine, f�r den Anhang wahrscheinlich besser geeignetere Variante.
%
%\lstinputlisting[caption={Ein 'Hallo Welt' Programm},label=code:hello_world,backgroundcolor=\color{white},frame=single,]{listings/Hello.c}
%
%\chapter{Aufbau eines minimalen \LaTeX{}-Dokuments}
%\lstinputlisting[breaklines=true,backgroundcolor=\color{white},numbers=none, language=TeX ]{listings/latex_example.tex}
%% mit \lstinputlisting lassen sich ganze quelltext-dateien in ein LaTeX-Dokument einbinden
%% die Optionen sind die gleichen wie bei \lstset