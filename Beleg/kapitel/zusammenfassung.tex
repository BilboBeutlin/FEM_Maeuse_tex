%-----------------------------------------------------------------------------
% kapitel/zusammenfassung.tex
%-----------------------------------------------------------------------------

\chapter{Zusammenfassung}

In der vorliegende Arbeit wurde das Hardwarelayout und die grundlegende Software f�r eine Schrittmotorsteuerung entwickelt. Das Resultat der Hardwareentwicklung ist in Abbildung 
 dargestellt. Es k�nnen Schrittmotoren mit maximalen Phasenstrom von $4,4\text{ A}$ betrieben werden. F�r eine Stromregelung mit h�heren Phasenstr�me m�ssen die Messwiderst�nde und Software angepasst werden. Die Schrittmotorsteuerung l�sst sich wie gew�nscht f�r die ben�tigten Motoren des Messstandes verwenden. Die Vorgabe, dass die Steuerung in einem Platinenrack untergebracht werden kann, wird als abschlie�ende Arbeit ausgef�hrt. 

Der Schwerpunkt der Arbeit lag neben dem Hardwareentwicklung haupts�chlich bei der Ansteuerung der Motoren und der umzusetzenden Stromregelung. Die Ergebnisse zeigen, dass sowohl Halbschritt- und Vollschrittbetrieb m�glich sind. Durch Einf�hrung einer Mittelwertbildung in der Stromregelung, konnte eine unempfindliche L�sung gegen�ber verrauschten Messsignalen und trotzdem mit hinreichender Dynamik umgesetzt werden.

%
			\begin{center}
				\captionsetup{type=figure}
				\includegraphics[width=1\linewidth]{bilder/lp.jpg}
				\caption{Fototgrafie der Schrittmotorsteuerung}
				\label{fig:lp}	
			\end{center}
			




%\begin{itemize}
	%\item Motivation, Einordnung, Umfeld und Abgrenzung der Arbeit
	%\item wesentliche Schwerpunkte und Ergebnisse der Arbeit
%\end{itemize}
%\vspace{0.5cm}
%
%\section*{Abstract}
%English Version.